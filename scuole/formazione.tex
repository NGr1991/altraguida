Scienze della Formazione
Presentazione
Scienze della Formazione conta circa 6000 studenti e offre due corsi di laurea triennali, quattro magistrali, ed uno magistrale a ciclo unico. I due corsi triennali e quello a ciclo unico sono a numero programmato ed hanno un test d'ingresso. 
Corsi di laurea triennale
     • Scienze dell'educazione (690 posti)
     • Comunicazione interculturale (300 posti) 
Il corso di laurea in Scienze dell'educazione prevede un tirocinio obbligatorio.
Corsi di laurea magistrale
     • Scienze pedagogiche 
     • Scienze antropologiche ed etnologiche 
     • Formazione e sviluppo delle risorse umane 
     • Psicologia dello sviluppo e dei processi educativi 
I corsi di laurea in Scienze pedagogiche e Formazione e sviluppo delle risorse umane prevedono un tirocinio obbligatorio.
Corsi di laurea magistrale a ciclo unico
     • Scienze della formazione primaria (400 posti) 
Il corso abilita alla professione di insegnante nelle scuole dell'infanzia e nella scuola primaria e prevede un tirocinio formativo ogni anno a partire dal secondo.
Studiare in Bicocca
La frequenza non è obbligatoria ma viene raccomandata, anche in ragione delle facilitazioni per i frequentanti, come sgravi in termini di carico di studio o l'ausilio di prove intermedie o preappelli. 
Contatti
Sito di facoltà: www.formazione.unimib.it 
I contatti dei rappresentanti si trovano nella sezione "persone", oppure al capitolo 7 di questa guida. 
Piattaforma e-learning, per studenti iscritti: http://formazione.elearning.unimib.it/ 
